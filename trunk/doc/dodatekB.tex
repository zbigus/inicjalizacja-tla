\chapter{Dodatek B - opis informatyczny programu}

W ramach niniejszej pracy zosta�y wykonane 2 programy napisane w j�zyku C++ przy u�yciu biblioteki OpenCV.

Pierwszy z nich \textbf{robust.exe} zawiera w sobie implementacj� 2 algorytm�w. S� nimi algorytm MOG oraz blokowy, kt�ry w zale�no�ci od parametr�w wej�ciowych u�ywa transformaty Hadamarda lub DCT. 

Program ten przyjmuje parametry wej�ciowe w nast�puj�cej kolejno�ci:

\begin{enumerate}
\item wielko�� bufora ramek
\item typ transformaty
\item nie wiem co wi�cej
\end{enumerate}

Drugi program \textbf{simple\_pixel\_methods.exe} zawiera w sobie implementacj� 3 algorytm�w. S� nimi algorytmy: �redniej z bufora, aproksymacji �redniej z bufora oraz mediany.

Program ten przyjmuje parametry wej�ciowe w nast�puj�cej kolejno�ci:

\begin{enumerate}
\item wielko�� bufora ramek
\item typ metody (0 - aproksymacja, 1 - �rednia, 2 - mediana)
\item warto�� parametru $\alpha$ w przypadku wybrania metody aproksymacji
\end{enumerate}

nie wiem co tu dalej napisa�...