\documentclass[pdflatex,11pt]{aghdpl}
% \documentclass{aghdpl}               % przy kompilacji programem latex
% \documentclass[pdflatex,en]{aghdpl}  % praca w j�zyku angielskim
\usepackage[polish]{babel}

\usepackage{polski}

\usepackage[cp1250]{inputenc}


% dodatkowe pakiety
\usepackage{enumerate}
\usepackage{listings}
\usepackage{mathtools}
\usepackage{bigstrut}
\usepackage{rotating}
\usepackage{multirow}
\usepackage{tocloft}
\usepackage{graphicx}
\usepackage{caption}
\usepackage{subcaption}
\usepackage{float}
\lstloadlanguages{TeX}

%\numberwithin{equation}{section}

%---------------------------------------------------------------------------

\author{Jakub Gola, Zbigniew Tekiela}
\shortauthor{J. Gola, Z. Tekiela}

\titlePL{Metody inicjalizacji modelu t�a}
\titleEN{Background model initialization methods}

\shorttitlePL{Metody inicjalizacji modelu t�a} % skr�cona wersja tytu�u je�li jest bardzo d�ugi
\shorttitleEN{Background model initialization methods}

\thesistypePL{Praca in�ynierska}
\thesistypeEN{Bachelor of Science Thesis}

\supervisorPL{dr hab. in�. Marek Gorgo�, prof. nadzw. AGH}
\supervisorEN{Marek Gorgo� Ph.D }

\date{2012}

\departmentPL{Katedra Automatyki i In�ynierii Biomedycznej}
\departmentEN{Department of Automatics and Biomedical Engineering}

\facultyPL{Wydzia� Elektrotechniki, Automatyki, Informatyki i In�ynierii Biomedycznej}
\facultyEN{Faculty of Electrical Engineering, Automatics, Computer Science and Biomedical Engineering}

\acknowledgements{Serdecznie dzi�kuj� \dots tu ci�g dalszych podzi�kowa� np. dla promotora, �ony, s�siada itp.}



\setlength{\cftsecnumwidth}{10mm}
\setcounter{tocdepth}{3}
\setcounter{secnumdepth}{4}
%---------------------------------------------------------------------------

\begin{document}

\titlepages

\tableofcontents
\clearpage

\chapter{Wst�p}
\label{wstep}

W dzisiejszych czasach systemy wizyjne znajduj� coraz wi�cej zastosowa� w~�yciu codziennym. Wraz ze wzrostem mocy obliczeniowej wsp�czesnych komputer�w oraz coraz ni�szych cen kamer i~sprz�tu wizyjnego mo�na zaobserwowa� dynamiczny rozw�j tej dziedziny informatyki. S� one u�ywane mi�dzy innymi w~systemach inteligentnego obszarowego sterowania ruchem, monitoringu przestrzeni miejskiej oraz w~bardziej zaawansowanych rozwi�zaniach CCTV, jak na przyk�ad systemy �ledzenia potencjalnych zagro�e� na lotniskach lub w~miejscach publicznych. We wszystkich wy�ej wymienionych zastosowaniach kluczow� kwesti� jest oddzielenie pierwszego planu od t�a. Odseparowanie dynamicznych element�w obrazu od statycznej scenerii jest podstaw� dzia�ania wszystkich algorytmy �ledz�cych.

Pomimo ci�g�ego rozwoju algorytm�w stosowanych do inicjalizacji modelu t�a nadal istniej� sytuacje w~kt�rych algorytmy te nie s� w~stanie wygenerowa� poprawnych rezultat�w. Najwi�ksze problemy sprawiaj� zmienne warunki o�wietlenia oraz drobne ruchy obiekt�w t�a (np. li�cie na wietrze). Kolejnym niepo��danym zjawiskiem jest wtapianie si� obiekt�w z~pierwszego planu w~t�o, gdy pozostaj� one przez d�ugi czas nieruchome. 
Nieustannie trwaj� prace nad stworzeniem algorytm�w eliminuj�cych lub minimalizuj�cych skutki wy�ej wymienionych zjawisk. Pr�by te zosta�y opisane mi�dzy innymi w~\cite{Baltieri2010} oraz \cite{Reddy2009}.
Metody te zaliczaj� si� do klasy metod przestrzennych (blokowych), a~ich zaimplementowanie i~por�wnanie b�dzie stanowi� jeden z~cel�w niniejszej pracy. Nast�pnie zostan� one skonfrontowane z~metodami operuj�cymi na poziomie pikseli - opisanymi w~\cite{7} oraz \cite{Wang2006}.


\chapter{Metody operuj�ce na poziomie pikseli}
\label{cha:Metody operuj�ce na poziomie pikseli}
Metody inicjalizacji modelu t�a dzia�aj�ce na poziomie pikseli dzia�aj� na najmniejszych mo�liwych cz�ciach obrazu, czyli na pojedynczych pikselach. Metody te bior� pod uwag� jedynie warto�� piksela w danym miejscu obrazu bez zwracania uwagi na bezpo�rednie s�siedztwo analizowanego elementu.
W~niniejszym rozdziale zostan� opisane 4 metody inicjalizacji t�a zaliczaj�ce si� do tej grupy:
\begin{itemize}
\item Mixture of Gaussians (MOG)
\item �rednia z bufora ramek
\item Aproksymacja �redniej przy u�yciu parametru alfa
\item Mediana z bufora ramek
\end{itemize}
\section{Mixture of Gaussians (MOG)}
\subsection{Wprowadzenie}
T�o sceny zawiera wiele dynamicznych obiekt�w jak poruszane na wietrze ga��zie i~li�cie drzew czy krzew�w. Zmiany w~obrazie nimi wywo�ane powoduj�, �e warto�� intensywno�ci danego piksela oraz jego kolor w~czasie mog� si� bardzo r�ni� od siebie. Z~tego powodu wykorzystanie pojedynczego przybli�enia rozk�adu prawdopodobie�stwa przy u�yciu krzywej Gaussa u�yte w \cite{Wren1997} daje z�e rezultaty. Zamiast tego w~opisywanej metodzie u�yto podej�cia bazuj�cego na wykorzystaniu kilku rozk�ad�w Gaussa o~r�nych parametrach w~celu zamodelowania takich zmian \cite{7}.

%Standardowy adaptacyjny model t�a polega na tworzeniu aproksymacji t�a, kt�re jest podobne do obecnej statycznej sceny za wyj�tkiem miejsc, w~kt�rych odby� si� ruch. Podej�cie takie jest efektywne w~sytuacjach, gdy obiekty poruszaj� si� nieustannie, a~t�o jest widoczne znaczn� cz�� czasu, jednak�e nie sprawdza si� ono dla scen zawieraj�cych du�o poruszaj�cych si� obiekt�w, a~w szczeg�lno�ci, gdy obiekty te poruszaj� si� powoli. Nie potrafi tak�e poradzi� sobie z~t�ami posiadaj�cymi rozk�ad dwumodalny (posiadaj�cy 2 najcz�ciej wyst�puj�ce warto�ci), powoli odtwarza t�o je�li zostanie ono odkryte i~ma jeden ustalony pr�g dla ca�ej sceny. 

W metodzie MOG zamiast modelowa� warto�ci wszystkich pikseli jako wy��cznie jeden typ rozk�adu, modelowane s� one mieszank� rozk�ad�w Gaussa. Bazuj�c na wadze i~wariancji ka�dego u�ytego do mieszanki rozk�adu Gaussa okre�lane jest, kt�re z~nich mog� odnosi� si� do pikseli t�a, a kt�re do obiekt�w pierwszego planu. Warto�ci pikseli, kt�re nie pasuj� do �adnego rozk�adu t�a uznawane s� za piksele nale��ce do pierwszego planu. Taki stan pikseli utrzymuje si� do czasu kiedy zaczn� wystarczaj�co dobrze pasowa� do kt�rego� z~rozk�ad�w t�a. 

Opisywana metoda dostosowuje si� tak, aby radzi� sobie ze zmianami o�wietlenia, powtarzalnymi ruchami element�w t�a, powolnie poruszaj�cymi si� elementami pierwszego planu oraz wprowadzaniu lub usuwaniu element�w ze sceny. Powolnie poruszaj�ce si� elementy potrzebuj� wi�cej czasu aby wpasowa� si� w~t�o, poniewa� rozk�ad do kt�rego pasuj� ma wi�ksz� wariancj� ni� t�o. Powtarzaj�ce si� zmiany r�wnie� s� uwzgl�dniane, a~model rozk�adu t�a jest utrzymywany nawet je�li jest chwilowo zamieniony przez inny rozk�ad, co prowadzi do szybszego odtworzenia t�a w~przypadku usuni�cia obiekt�w ze sceny.
W wi�kszo�ci implementacji tej metody inicjalizacja obrazu odbywa si� za pomoc� pierwszej ramki.
%Metoda ta wykorzystuje dwa podstawowe parametry: 
%\begin{lstlisting}
%a - sta�a uczenia
%T - porcja danych jaka powinna by� uwzgl�dniona w~tle
%\end{lstlisting}

%Zak�adaj�c, �e ka�dy piksel obrazu b�dzie pochodzi� z~jednej p�aszczyzny przy zmiennym �wietleniu, to do wydzielenia t�a z~takiego obrazu wystarczy u�ycie pojedynczej adaptacyjnej aproksymacji rozk�adem Gaussa dla ka�dego piksela. Jednak�e w~rzeczywistych warunkach na obrazie wyst�puje wiele powierzchni oraz zmienne o�wietlenie, co sprawia, �e potrzebne staje si� u�ycie kilku adaptacyjnych rozk�ad�w Gaussa. W~opisywanej metodzie u�yta jest mieszanka kilku takich rozk�ad�w o~r�nych parametrach. Za ka�dym razem, gdy parametry rozk�adu s� aktualizowane, nast�puje proces oceny dost�pnych rozk�ad�w w~celu okre�lenia tego najbardziej prawdopodobnego. 

\subsection{Opis metody}
\subsubsection{Model mieszanki}

Niech kolejne warto�ci danego piksela w~czasie nazywaj� si� histori� piksela. Zatem jest to ci�g warto�ci danego piksela, gdzie dla obraz�w w~skali szaro�ci s� to warto�ci skalarne, a~dla obraz�w kolorowych s� to wektory. O~danym pikselu $\{x_0,y_0\}$ w~danej chwili czasu t, mo�na powiedzie�, �e znana jest jego historia
\begin{equation}
\{X_1,\dots,X_t\}=\{I(x_0,y_0,i):1\le i \le t\}
\end{equation}
gdzie: $I$ jest sekwencj� ramek.

Niedawna historia dla ka�dego piksela modelowana jest mieszank� K~rozk�ad�w Gaussa. Prawdopodobie�stwo zaobserwowania obecnego piksela okre�la si� wzorem

\begin{equation}
P(X_t)=\sum_{i=1}^{K} \omega_{i,t} \cdot \eta(X_t, \mu_{i,t} , \Sigma_{i,t})
\end{equation}

\noindent gdzie K~jest liczb� rozk�ad�w,  $\omega_{i,t}$ jest oszacowan� wag� $i$-tego rozk�adu w~mieszance w~chwili $t$, $\mu_{i,t}$ jest �redni� warto�ci�  $i$-tego rozk�adu w~mieszance w~chwili $t$, $\Sigma_{i,t}$ jest macierz� kowariancji w~chwili $t$,$\eta$ jest funkcj� g�sto�ci prawdopodobie�stwa Gaussa:

\begin{equation}
\eta(X_t,\mu,\Sigma)=\frac{1}{(2\pi)^\frac{n}{2} |\Sigma|^\frac{1}{2}}e^{-\frac{1}{2}(X_t-\mu_t)^T \Sigma^{-1} (W_t -\mu_t)}
\end{equation}

\noindent gdzie: K~okre�lane jest przez dost�pn� pami�� oraz moc obliczeniow� jednostki na kt�rej wykonywany jest algorytm. Najcz�ciej spotykana warto�� mie�ci si� w~zakresie od 3~do 5. W~celu zmniejszenia ilo�ci oblicze� przyjmuje si�, �e macierz kowariancji ma posta�:

\begin{equation}
\Sigma_{k,t}=\sigma^2_kI
\end{equation}

Takie rozwi�zanie zak�ada, �e warto�ci dla poszczeg�lnych sk�adowych barwnych ka�dego piksela maj� tak� sam� wariancj�. Za�o�enie to obci��one jest pewnymi b��dami, lecz pozwala unikn�� wykonywania operacji odwracania macierzy, co jest zadaniem bardzo kosztownym, za cen� mniejszej precyzji. 

Zatem rozk�ad ostatnio obserwowanych warto�ci dla ka�dego piksela w~obrazie opisany jest mieszank� rozk�ad�w Gaussa. Nowa warto�� piksela b�dzie na og� reprezentowana przez jeden z~g��wnych sk�adnik�w mieszanki. 

Do obliczenia warto�ci nowego piksela u�yta zosta�a metoda aproksymacji K �rednimi. Ka�da warto�� piksela jest sprawdzana pod k�tem dopasowania do jednego z~K rozk�ad�w. Obliczana jest odleg�o�� piksela od ka�dego z rozk�ad�w, zdefiniowana jako 2,5 odchylenia standardowego z~danego rozk�adu. Opisany spos�b wyboru odpowiednich pikseli jest bardzo u�yteczny dla obszar�w z~r�nym o�wietleniem, poniewa� obiekty znajduj�ce si� w~zacienionych obszarach maj� mniejszy szum ni� obiekty znajduj�ce si� w~ja�niejszych regionach.

W przypadku, gdy �aden rozk�ad nie zosta� dopasowany do danego piksela, to rozk�ad z~najmniejsz� wag� jest zast�powany nowym z~warto�ci� piksela jako now� warto�ci� oczekiwan�, du�� wariancj� i~nisk� wag�.

Wcze�niejsze wagi K~rozk�ad�w w~czasie t~s� aktualizowane wg wzoru: 

\begin{equation}
\omega_{k,t}=(1-\alpha)\omega_{k,t-1}+\alpha(M_{k,t})
\end{equation}
gdzie: $\alpha$ to tempo uczenia, a~$M_{k,t}$ jest 1~dla dopasowanego rozk�adu i~0 dla pozosta�ych rozk�ad�w. Po dokonaniu aproksymacji wagi s� normalizowane. $\frac{1}{\alpha}$ oznacza sta�� czasow� okre�laj�c� pr�dko�� z~jak� parametry rozk�ad�w si� zmieniaj�.

Dla rozk�ad�w niedopasowanych do danego piksela warto�ci $\mu$ i~$\sigma$ pozostaj� niezmienione. Dla rozk�ad�w, kt�re zosta�y dopasowane warto�ci $\mu$ i~$\sigma$ obliczane s� nast�puj�co:

\begin{equation}
\mu_t=(1-\rho)\mu_{t-1}+\rho X_t
\end{equation}

\begin{equation}
\sigma^2_t=(1-\rho)\sigma^2_{t-1}+\rho(X_t-\mu_t)^T(X_t-\mu_t)
\end{equation}
przy czym:

\begin{equation}
\rho = \alpha\eta(X_t|\mu_k,\sigma_k)
\end{equation}

Jedn� z~najwi�kszych zalet opisanego rozwi�zania jest fakt, �e kiedy jaki� obiekt zostanie dodany do t�a, to nie niszczy on istniej�cego modelu t�a. Oryginalny kolor t�a zostaje zachowany w~mieszance do czasu, a� stanie si� najmniej prawdopodobnym kandydatem oraz zostanie zaobserwowany nowy kolor. Zatem je�li obiekt pozostanie nieruchomy wystarczaj�co d�ugo aby sta� si� cz�ci� t�a, a~nast�pnie si� poruszy, to rozk�ad opisuj�cy poprzednie t�o ci�gle istnieje w~tymi samymi $\mu$ i~$\sigma^2$, ale mniejszym $\omega$ przez co mo�e zosta� szybko ponownie do��czony do modelu t�a.

\subsubsection{Estymacja modelu t�a}


Podczas gdy parametry modelu mieszanki dla ka�dego piksela zmieniaj� si�, nale�y okre�li� kt�re rozk�ady z~mieszanki daj� najwi�ksze prawdopodobie�stwo bycia wygenerowanymi przez procesy t�a. Z~heurystycznego punktu widzenia najbardziej interesuj�ce s� rozk�ady, kt�re daj� najlepsze dopasowanie i~najmniejsz� wariancj�


W celu wybrania odpowiednich rozk�ad�w nale�y uszeregowa� je wed�ug warto�ci $\omega$/$\sigma$. Warto�� ta ro�nie zar�wno przy zwi�kszeniu dopasowania, jak i~przy zmniejszeniu wariancji. W~praktyce tak uszeregowana lista daje zbi�r rozk�ad�w, gdzie najbardziej prawdopodobni kandydaci znajduj� si� na pocz�tku, a~najmniej prawdopodobni na ko�cu.
Selekcji pierwszych rozk�ad�w do modelu t�a dokonuje si� za pomoc� wzoru 

\begin{equation}
B = argmin_b\left(\sum_{k=1}^{b}\omega_k>T\right)
\end{equation}
gdzie: T~jest miar� minimalnej ilo�ci danych jaka powinna by� prana pod uwag�. Rozwi�zanie takie bierze pod uwag� najlepiej dostosowany rozk�ad dot�d, a� pewna porcja T~danych jest rozwa�ona. Je�li T~jest jest warto�ci� ma��, to wtedy model t�a zazwyczaj jest unimodalny. Je�li T~jest warto�ci� wi�ksz�, multimodalny rozk�ad wywo�any powtarzalnymi ruchami w~tle mo�e skutkowa� uwzgl�dnieniem w~modelu t�a wi�cej ni� jednego koloru. Rezultatem tego jest efekt przezroczysto�ci, kt�ry pozwala modelowi przyjmowa� dwa lub wi�cej oddzielnych kolor�w. 


\section{�rednia z~bufora ramek}
\subsection{Wprowadzenie}
�rednia z~bufora ramek jest jedn� z~najprostszych mo�liwych metod inicjalizacji t�a. W tej metodzie t�o inicjalizowane jest na podstawie ca�ego bufora ramek. Algorytm ten opiera si� na wyliczaniu �redniej arytmetycznej dla ka�dej pozycji piksela na obrazie spo�r�d ramek zgromadzonych w~buforze.

\subsection{Opis metody}
Warto�� dla ka�dego piksela t�a wyliczana jest ze wzoru.

\begin{equation}
B(x,y)=\frac{1}{n} \sum_{i=1}^{n} P(x_i,y_i)
\end{equation}

\noindent gdzie: $B(x,y)$ oznacza obecn� warto�� piksela t�a w~miejscu (x,y), $n$ oznacza rozmiar bufora, $P(x_i,y_i)$ oznacza warto�� piksela na pozycji (x,y) w~$i$-tej ramce w~buforze.


\section{Aproksymacja �redniej przy u�yciu parametru alfa}
\subsection{Wprowadzenie}

Metoda ta jest modyfikacj� �redniej z~bufora ramek. Jej g��wnym za�o�eniem jest aproksymacja �redniej za pomoc� morfingu ramek. Daje ona podobne rezultaty, jednak�e w~przeciwie�stwie do swojego pierwowzoru nie wykorzystuje bufora, a przez to charakteryzuje si� mniejsz� z�o�ono�ci� pami�ciow�. Inicjalizacja t�a przeprowadzana jest na podstawie pierwszej ramki.

\subsection{Opis metody}
Warto�� dla ka�dego piksela t�a wyliczana jest ze wzoru:

\begin{equation}
\label{alpha}
B_{n}(x,y) = P_{n}(x,y) \cdot (1-\alpha) + B_{n-1}(x,y) \cdot \alpha
\end{equation}

\noindent gdzie: $B_{n}(x,y)$ oznacza obecn� warto�� piksela t�a w~miejscu (x,y), $P_{n}(x,y)$ oznacza warto�� piksela na pozycji (x,y) w~obecnej ramce, $B_{n-1}$ oznacza warto�� piksela t�a na pozycji (x,y) w~poprzedniej iteracji algorytmu.

\section{Mediana z bufora ramek}
\subsection{Wprowadzenie}

Mediana z~bufora ramek jest kolejn� z prostych metod inicjalizacji t�a. Do zainicjalizowania t�a u�ywa ona ca�ego bufora ramek. Algorytm ten opiera si� na wyliczaniu mediany dla ka�dej pozycji piksela na obrazie spo�r�d ramek zgromadzonych w~buforze.

\subsection{Opis metody}
Warto�� dla ka�dego piksela t�a wyliczana jest ze wzoru.

\begin{equation}
\label{alpha}
B(x,y) = \widetilde{P(x,y)}, \; gdzie \;  P(x,y) = \{P_{1}(x,y), P_{2}(x,y), \dots , P_{n}(x,y)\}
\end{equation}

\noindent gdzie: $B(x,y)$ oznacza warto�� piksela t�a w~miejscu (x,y), $P(x,y)$ oznacza zbi�r warto�� piksela na pozycji (x,y) we wszystkich ramkach w buforze, $\widetilde{P(x,y)}$ oznacza median� (warto�� �rodkow�) na pozycji (x,y) we wszystkich ramkach w buforze, $P_{i}(x,y)$ oznacza warto�� piksela na pozycji (x,y) \newline w $i$-tej ramce, gdzie $i \in \{1,2,\dots,n\}$.

\chapter{Metody blokowe (przestrzenne)}
\label{cha:metodyprzestrzenne}
\section{Wprowadzenie}
\label{blo:wstep}
Metody przestrzenne, w~przeciwie�stwie do punktowych, rozwa�aj� ramk� wideo jako grup� blok�w o~okre�lonym rozmiarze oraz wsp�rz�dnych, lokalizuj�cych jednoznacznie dany blok w~danej ramce. 
Za�o�enie tej klasy algorytm�w polega na przypuszczeniu, i� blok kt�ry pojawia si� najcz�ciej w~danej sekwencji wideo jest najlepszym kandydatem na bycie fragmentem t�a. Bloki o~tych samych wsp�rz�dnych s� por�wnywane mi�dzy sob� w~kolejnych ramkach sekwencji wideo (np. przy u�yciu wsp�czynnika korelacji poszczeg�lnych pikseli) w celu utworzenia grupy kandydat�w - zbioru blok�w, kt�re z pewnym prawdopodobie�stwem nalez� do t�a i na podstawie kt�rego (z u�yciem odpowiednich transformat) t�o to zostanie odtworzone. 
Proces ten mo�na zilustrowa� nast�puj�cym schematem blokowym:
TU WSTAWI� SCHEMAT !!!!!!
W dalszej cz�ci tego rozdzia�u zostan� om�wione szczeg�owe zasady dzia�ania tej rodziny algorytm�w, jak i~r�wnie� zostan� przedstawione dwie metody, stosowane do wyboru najlepszego bloku z~grupy kandydat�w, wykorzystuj�ce odpowiednio transformat� DCT oraz rekursywn� transformat� Hadamarda.
\section{Algorytm dzia�ania}
\label{blo:algorytm}
Podstawow� jednostk�, na kt�rej operuje ta klasa algorytm�w, jest blok. Ka�dy blok jest okre�lony przez jego wag� oraz piksele, kt�re zawiera. Dodatkowo, dla ca�ej sekwencji, dla ka�dej lokalizacji blokowej (czyli miejsca na obrazie, gdzie znajduj� si� bloki o~okre�lonych wsp�rz�dnych) jest utrzymywana tzw. grupa kandydat�w - blok�w, kt�re z~pewnym prawdopodobie�stwem nale�� do t�a. Waga danego bloku z~tej grupy okre�la, jak cz�sto w~sekwencji wideo pojawia� si� blok podobny do niego (tzn. spe�niaj�cy okre�lone kryteria, om�wione w~sekcji \ref{blo:kandydaci})

Za \cite{Reddy2009} przyj�to nast�puj�ce oznaczenia, u�ywane w~dalszej cz�ci niniejszej pracy:
\begin{itemize}
\item $W$,$H$ - odpowiednio szeroko�� i~wysoko�� ramki,
\item $I_f$ - ramka nr f,
\item $B_f(i,j)$ - blok ramki f~o~wsp�rz�dnych (i,j),
\item $b_f(i,j)$ blok $B_f(i,j)$ po wektoryzacji ,
\item $R(i,j)$ -zbi�r kandydat�w (grupa kandydat�w) dla lokalizacji blokowej o~wsp�rz�dnych $(i,j)$,
\item $r_k(i,j)$ - k-ty blok z~grupy kandydat�w  $R(i,j)$ po wektoryzacji,
\item $W_k(i,j)$ - waga k-tego bloku z~grupy kandydat�w  $R(i,j)$,
\item $\mu_{r_k}$, $\mu{b_f}$ - �rednia z~element�w wektor�w blokowych odpowiednio $r_k$ oraz $b_f$,
\item $\sigma_{r_k}$, $\sigma{b_f}$ - odchylenie standardowe z~element�w wektor�w blokowych odpowiednio $r_k$ oraz $b_f$.
\end{itemize}
Ka�d� z~opisywanych metod mo�na podzieli� na trzy zasadnicze fazy: 
\begin{enumerate}
\item Kolekcjonowanie kandydat�w - blok�w, kt�re mog� si� zawiera� w~tle.
\item Cz�ciow� rekonstrukcj� t�a.
\item Estymacj� t�a na podstawie grup kandydat�w.
\end{enumerate}
W kolejnych sekcjach ka�da z~faz zostanie szczeg�owo om�wiona. 

\subsection{Kolekcjonowanie kandydat�w}
\label{blo:kandydaci}
W tej fazie nast�puje obr�bka ka�dej kolejnej ramki sekwencji wideo. Ka�da ramka $f$ jest dzielona na bloki $B_f(i,j)$ o~rozmiarze $N\cdot N$ ka�dy. Nast�pnie ka�dy blok $B_f(i,j)$ jest poddawany procesowi wektoryzacji, tj. zamieniany na wektor $b_f(i,j)$ o d�ugo�ci $N^2$ -tzw. wektor blokowy -   poprzez ��czenie ze sob� kolejnych wierszy  W kolejnym kroku tej fazy algorytmu, ka�dy wektor blokowy ($b_f(i,j)$ jest por�wnywany z~ka�dym wektorem blokowym $r_k(i,j)$ znajduj�cym si� w~grupie kandydat�w dla danej lokalizacji blokowej ($R(i,j)$). Je�li nie jest podobny do �adnego z~kandydat�w, zostaje dodany jako nowy kandydat z~pocz�tkow� wag� r�wn� jeden. W~przeciwnym wypadku, ka�dy podobny wektor blokowy i~jego waga s� aktualizowane wg nast�puj�cych wzor�w: 
\begin{equation}
r_k(i,j)=\frac{r_k(i,j)W_k(i,j)+b_f(i,j)}{W_k(i,j)+1} 
\end{equation}
\begin{equation}
W_k(i,j)=W_k(i,j)+1
\end{equation}
\subsubsection{Kryteria podobie�stwa blok�w}
Kluczowe dla prawid�owego dzia�ania algorytmu jest dobranie odpowiednich kryteri�w podobie�stwa blok�w (a raczej wektor�w blokowych, gdy� na nich wykonywane s� obliczenia). Najcze�ciej w~tym celu u�ywa si� wsp�czynnika korelacji oraz wsp�czynnika MAD\footnote{MAD - Mean of Absolute Differences}. S� one wyliczane nastepuj�co:
\begin{equation}
\label{Tcorr}
T_{corr}=\frac{{(r_k(i,j)-\mu_{r_k}(i,j))}^T(b_f(i,j)-\mu_{b_f}(i,j))}{\sigma_{r_k}(i,j)\sigma_{b_f}(i,j)}
\end{equation}
\begin{equation}
T_{MAD}=\frac{1}{N^2}\sum_{n=0}^{N^2-1}\left|b_{f_n}(i,j)-r_{k_f}(i,j)\right|
\end{equation}

Wsp�czynnik korelacji $T_{corr}$ odpowiada za podobie�stwo wektor�w blokowych mi�dzy sob� i~zazwyczaj wymaga si�, aby by� powy�ej pewnej warto�ci (np. 0.8) w~celu uznania wektor�w za podobne. Jednak�e cz�sto bywa on niewystarczaj�cy, gdy� mo�e zdarzy� si�, i� dwa ca�kowicie niepodobne wektory blokowe b�d� mia�y bardzo wysoki wsp�czynnik korelacji. W~celu minimalizacji tego zjawiska wprowadzono wsp�czynnik $T_{mad}$, kt�ry musi by� odpowiednio niski dla dw�ch wektor�w, by zosta�y uznane za podobne. 
\subsection{Cz�ciowa rekonstrukcja t�a}
Po wyliczeniu kandydat�w dla wszystkich lokalizacji blokowych nast�puje cz�ciowa rekonstrukcja t�a. Polega ona na znalezieniu wszystkich lokalizacji blokowych maj�cych tylko jednego kandydata i~przypisaniu ich warto�ci do odpowiednich blok�w rekonstruowanego t�a (po wcze�niejszym odtworzeniu blok�w z wektor�w blokowych). 
\subsection{Estymacja brakuj�cego t�a}
W celu przeprowadzenia estymacji t�a zosta�o wprowadzone poj�cie superbloku, definiowanego jako klaster o~wymiarach $2 \cdot 2$ bloki.
\begin{figure}[ht]
\begin{center}
\includegraphics{superblok.png}
\caption{Blok X~oraz jego 8-punktowe otoczenie}
\end{center}
\end{figure}

Przyk�adowo, dla bloku X~z~powy�szego rysunku mo�emy wyr�ni� nast�puj�ce superbloki: \{B,C,A,X\}, \{C,D,X,E\}, \{A,X,H,G\} oraz \{X,E,G,F\}.
W tej fazie algorytmu dla ka�dego superbloku, kt�ry zawiera 3~bloki wype�nione t�em, jest szacowany czwarty, brakuj�cy blok.Estymacja odbywa si� w~dziedzinie cz�stotliwo�ci. Ka�da z~omawianych metod skupia si� na analizie wysokich cz�stotliwo�ci, gdy� to one odpowiadaj� za zmienno�� obrazu. Faza ko�czy si� gdy ca�e t�o zostanie zrekonstruowane.
Omawiane dwa algorytmy r�ni� si� doborem transformaty, a ka�dy z nich operuje na blokach odtwarzanych z odpowiadaj�cych wektor�w blokowych.
\subsubsection{Estymacja z~wykorzystaniem DCT}
 Metoda ta zosta�a zaproponowana w~\cite{Reddy2009}. Dla ka�dego superbloku s� w niej tworzone dwie r�ne wersje transformaty.
 \begin{enumerate}
 \item blok X~jest zerowany, natomiast brana jest pod uwag� zawarto�� s�siednich blok�w. Na superbloku jest przeprowadzana dwuwymiarowa dyskretna transformata kosinusowa, a~jej wsp�czynniki s� zapisywane w~macierzy $C$ o~wymiarach $M \cdot M$. Wsp�czynnik DC macierzy C~(o wsp�rz�dnych (0,0)) jest ustawiany na 0, przez co pod uwag� zostanie wzi�te tylko przestrzenne zr�nicowanie warto�ci poszczeg�lnych pikseli. 
 \item bloki otaczaj�ce X~s� zerowane, natomiast X~jest inicjalizowany kolejnymi warto�ciami $r_k$ ze zbioru kandydat�w $R$ dla danej lokalizacji blokowej. Powstaje zatem $k$~wersji superbloku. Na ka�dym z~superblok�w jest przeprowadzana 2D DCT, a~jej wsp�czynniki s� zachowywane w~macierzy $D_k$, gdzie k~to numer kolejnego bloku ze zbioru kandydat�w. Tak jak poprzednio, wsp�czynnik DC macierzy $D_k$ jest ustawiany na zero. 
 \end{enumerate}
  
  Nale�y zauwa�y�, i� w~wyniku zastosowania dw�ch przeciwstawnych masek superbloku (tj. zerowania okre�lonych blok�w do niego nale��cych) warto�ci pikseli w~obszarze wysokich cz�stotliwo�ci b�d� przeciwstawne w~macierzach $C$ oraz $D_k$, przez co zredukuj� si� one przy dodawaniu. Istniej� jednak przypadki, gdy tak si� nie dzieje i~w~macierzach $C$ oraz $D_k$ nie ma element�w o~wysokich cz�stotliwo�ciach - zdarza si� to gdy warto�ci pikseli w~niewyzerowanych blokach sa bliskie zeru. Aby temu zapobiec, analizuje si� �redni� pikseli $\mu_k$ bloku $r_k$ - je�li jest wy�sza lub r�wna 128, odpowiednie bloki w~obu wersjach superbloku s� zerowane, a~je�li �rednia jest ni�sza - wszystkie piksele tych blok�w s� ustawiane na 255. Korekta ta zapewnia, �e w~ka�dym superbloku obszar wysokich cz�stotliwo�ci nie b�dzie pusty.
  
  W�asno�� redukowania si� wysokich cz�stotliwo�ci przy dodawaniu dw�ch komplementarnych superblok�w wykorzystano przy tworzeniu funkcji kosztu, wyznaczaj�cej najlepszy blok do uzupe�nienia brakuj�cego t�a. 
  Funkcja ta ma nast�puj�c� posta� 
  \begin{equation}
  \label{costfun}
  cost(k)=\left(\sum_{v=0}^{M-1}\sum_{u=0}^{M-1}\left|C(v,u)+D_k(v,u)\right|\right)\lambda_k
  \end{equation}
  \begin{equation}
  \lambda_k=e^{-\alpha\omega_k}
  \end{equation}
  
  gdzie: $a\in\langle0,1\rangle$, $\omega_k=\frac{W_k}{\sum_{k=0}^{L-1}W_k}$, przy czym $W_k$ jest wag� elementu $r_k$, a~L~jest liczb� element�w zbioru $r_k$. Wsp�czynnik $\alpha$ jest dobierany zazwyczaj eksperymentalnie i~okre�la, jak du�y wp�yw na wynik funkcji kosztu ma waga danego bloku (czyli tak naprawd� cz�sto�� jego wyst�powania w~sekwencji wideo). 
  Blok o~najni�szej warto�ci funkcji kosztu (czyli taki, kt�ry najlepiej redukuje sum� wysokich cz�stotliwo�ci w~macierzach C~i~D~) zostaje dodany jako najbardziej wiarygodna estymacja t�a.
   \subsubsection{Estymacja z~wykorzystaniem rekursywnej transformaty Hadamarda}
   Metoda zosta�a przedstawiona po raz pierwszy w~pracy \cite{Baltieri2010} Ka�da ramka jest w niej dzielona na bloki o~rozmiarze $16 \cdot 16$ pikseli. Do przeprowadzenia trzeciego etapu tej metody jest wykorzystywana dyskretna transformata Hadamarda, b�d�ca generalizacj� transformaty Fouriera. Opiera si� ona na macierzach Hadamarda H, definiowanych rekursywnie w~nast�puj�cy spos�b:
   \begin{equation}
    H_1=\left[1\right] \\
    \end{equation}
    \begin{equation}
   H_{2N}=\begin{bmatrix}
   H_N&H_N \\
   H_N&-H_N \\
   \end{bmatrix}
   \end{equation}
   
   Transformat� F~bloku $X$ o~wymiarach $2N \cdot 2N$ wyra�a si� jako
   \begin{equation}
   F=MXM
   \end{equation}
   gdzie: M~to macierz Hadamarda rz�du 2N. Bardzo u�yteczn� w�asno�ci� tej transformaty jest mo�liwo�� rozbicia jej na sum� transformat rz�du ni�szego. Przyk�adowo, po rozbiciu bloku X~na podbloki A,B,C,D o~wymiarach $N\cdot N$  transformat� mo�na obliczy� ze wzoru
   \begin{equation}F=
   \begin{bmatrix}
   H& H\\
   H & -H
   \end{bmatrix}
   \begin{bmatrix}
   A& B\\
   C& D
   \end{bmatrix}
      \begin{bmatrix}
      H& H\\
      H & -H
      \end{bmatrix}
   \end{equation}
gdzie: H~to macierz Hadamarda rz�du N.   
Wyra�aj�c macierz F~jako:
\begin{equation}
\label{matrixf}
F=
\begin{bmatrix}
f_{1,1} & f_{1,2}\\
f_{2,1} & f_{2,2}
\end{bmatrix}
\end{equation}
otrzymuje si�:
\begin{equation}
\label{matrixfelements}
\begin{cases} 
f_{1,1}=HAH+HBH+HCH+HDH\\ f_{1,2}=HAH-HBH+HCH-HDH\\
f_{2,1}=HAH+HBH-HCH-HDH\\ 
f_{2,2}=HAH-HBH-HCH+HDH\\ 
\end{cases}
\end{equation}
Stosuj�c wzory (\ref{matrixf}) oraz (\ref{matrixfelements}) mo�na obliczy� macierz Hadamarda rz�du $2N$ z~czterech macierzy rz�du $N$, co b�dzie pomocne w~przy wyliczaniu transformat superblok�w.
Tak samo jak w~metodzie DCT, dla ka�dego superbloku tworzone s� 2~wersje transformat: pierwsza - z~wyzerowanym blokiem X, bior�ca pod uwag� zawarto�� blok�w s�siednich oraz druga - z~wyzerowanymi s�siednimi blokami, bior�ca pod uwag� tylko zawarto�� bloku X. Funkcja kosztu, okre�laj�ca kt�ry blok jest najlepszym kandydatem do bycia blokiem t�a, jest identyczna jak we wzorze \ref{costfun}. 
%Algorytm wprowadza r�wnie� poprawk� na integralno�� wybranego bloku z~reszt� t�a - je�li jego �redni gradient wzd�u� przynajmniej dw�ch kraw�dzi jest wi�kszy ni� $\gamma$, blok jest odrzucany, po czym nast�puje analiza kolejnego bloku z~grupy kandydat�w z~najmniejsz� warto�ci� funkcji kosztu.
%Wsp�czynnik $\gamma$ jest dobierany eksperymentalnie.   
\chapter{Metodologia bada�}
\section{U�yte sekwencje}
W niniejszej pracy do oceny jako�ci algorytm�w inicjalizacji t�a u�yto pierwszych 200 ramek nast�puj�cych sekwencji video:
\begin{enumerate}
\item clip\_01.mpg - sekwencja zarejestrowana przez kamer� umieszczon� na �cianie budynku C3 AGH w pochmurny dzie�, przedstawiaj�ca ludzi chodz�cych po chodniku oraz cz�� jezdni ulicy Czarnowiejskiej wraz z poruszaj�cymi si� na niej samochodami
\item clip\_02.mpg - jak w pkt. 1, lecz sekwencja jest w gorszej jako�ci
\item clip\_03.mpg - jak w pkt. 1, lecz przy zmiennych warunkach o�wietleniowych
\item clip\_04.mpg - jak w pkt. 1, lecz ze sta�ym cieniem
\item fountain.mpg - sekwencja przedstawiaj�ca na pierwszym planie dzia�aj�c� fontann� na tle parkingu z poruszaj�cymi si� samochodami
\item highway.mpg - sekwencja zarejestrowana przez kamer� umieszczon� nad autostrad�
\item pedestrians.mpg  - sekwencja zarejestrowana przez kamer� umieszczon� na dworcu kolejowym, przedstawiaj�ca poruszaj�cych si� po peronie pasa�er�w
\item sidewalk.mpg - sekwencja zarejestrowana przez kamer� umieszczon� nad przej�ciem dla pieszych; na obrazie wyst�puj� mocne drgania kamery.
\end{enumerate}

\section{U�yte metryki}



\chapter{Badania}
\label{badania}
Dla ka�dej z u�ytych sekwencji wideo zosta�y policzone wszystkie metryki wymienione w rozdziale~4. 
Poni�ej, dla ka�dej metody, przedstawiony zostanie obraz wynikowy dla dw�ch przypadk�w: najgorszych oraz najlepszych uzyskanych metryk: MSE - b��du �redniokwadratowego oraz PP - Procentu podobie�stwa o zadanym progu.

\section{Metoda �redniej z bufora ramek}

Dla tej metody najlepsze warto�ci metryk zosta�y uzyskane dla sekwencji clip\_02.mpg i wynosz� one odpowiednio:

\begin{table}[htbp]
  \centering
    \begin{tabular}{|r|r|}
    \hline
    MSE   & 38,39342287 \\
    \hline
    PP(80\%) & 0,999278003 \\
    \hline
    PP(90\%) & 0,977152724 \\
    \hline
    PP(98\%) & 0,762080394 \\
    \hline
    PP(99\%) & 0,529932411 \\
    \hline
    \end{tabular}%
 \caption{Warto�ci metryk dla sekwencji clip\_02.mpg}
\end{table}%


Poni�ej przedstawiono obrazy: referencyjny, wynikowy oraz ich r�nic� dla tej sekwencji.


\begin{figure}[ht]
        \centering
        \begin{subfigure}[b]{0.32\textwidth}
                \centering
                \includegraphics[width=\textwidth]{clip_02_ref_gray.png}
                \caption{Obraz referencyjny}
        \end{subfigure}
        \begin{subfigure}[b]{0.32\textwidth}
                \centering
                \includegraphics[width=\textwidth]{bg_clip_02mpg_mean_k_200.png}
                \caption{Obraz wynikowy}
        \end{subfigure}
        \begin{subfigure}[b]{0.32\textwidth}
                \centering
                \includegraphics[width=\textwidth]{bg_clip_02mpg_mean_k_200.png}
                \caption{Obraz r�nicowy}
        \end{subfigure}
        \caption{Obrazy wynikowe dla sekwencji clip\_02.mpg}
\end{figure}

\newpage
Najgorsze warto�ci metryk zosta�y uzyskane dla sekwencji clip\_03.mpg i wynosz� one odpowiednio:


\begin{table}[htbp]
  \centering
    \begin{tabular}{|r|r|}
    \hline
    MSE   & 198,5494437 \\
    \hline
    PP(80\%) & 0,98932228 \\
    \hline
    PP(90\%) & 0,847793561 \\
    \hline
    PP(98\%) & 0,093102758 \\
    \hline
    PP(99\%) & 0,038605068 \\
    \hline
    \end{tabular}%
  \caption{Warto�ci metryk dla sekwencji clip\_03.mpg}
\end{table}


Poni�ej przedstawiono obrazy: referencyjny, wynikowy oraz ich r�nic� dla tej sekwencji.


\begin{figure}[ht]
        \centering
        \begin{subfigure}[b]{0.32\textwidth}
                \centering
                \includegraphics[width=\textwidth]{clip_03_ref_gray.png}
                \caption{Obraz referencyjny}
        \end{subfigure}
        \begin{subfigure}[b]{0.32\textwidth}
                \centering
                \includegraphics[width=\textwidth]{bg_clip_03mpg_mean_k_200.png}
                \caption{Obraz wynikowy}
        \end{subfigure}
        \begin{subfigure}[b]{0.32\textwidth}
                \centering
                \includegraphics[width=\textwidth]{bg_clip_03mpg_mean_k_200.png}
                \caption{Obraz r�nicowy}
        \end{subfigure}
        \caption{Obrazy wynikowe dla sekwencji clip\_03.mpg}
\end{figure}


%------------------------------------------------------------------------------
\section{Metoda mediany z bufora ramek}

Dla tej metody najlepsze warto�ci metryk zosta�y uzyskane dla sekwencji clip\_02.mpg i wynosz� one odpowiednio:

\begin{table}[htbp]
  \centering
    \begin{tabular}{|r|r|}
    \hline
    MSE   & 13,15359824 \bigstrut\\
    \hline
    PP(80\%) & 0,998912656 \bigstrut\\
    \hline
    PP(90\%) & 0,996189946 \bigstrut\\
    \hline
    PP(98\%) & 0,894175315 \bigstrut\\
    \hline
    PP(99\%) & 0,638897346 \bigstrut\\
    \hline
    \end{tabular}%
 \caption{Warto�ci metryk dla sekwencji clip\_02.mpg}
\end{table}%

Poni�ej przedstawiono obrazy: referencyjny, wynikowy oraz ich r�nic� dla tej sekwencji.


\begin{figure}[ht]
        \centering
        \begin{subfigure}[b]{0.32\textwidth}
                \centering
                \includegraphics[width=\textwidth]{clip_02_ref_gray.png}
                \caption{Obraz referencyjny}
        \end{subfigure}
        \begin{subfigure}[b]{0.32\textwidth}
                \centering
                \includegraphics[width=\textwidth]{bg_clip_02mpg_median_k_200.png}
                \caption{Obraz wynikowy}
        \end{subfigure}
        \begin{subfigure}[b]{0.32\textwidth}
                \centering
                \includegraphics[width=\textwidth]{bg_clip_02mpg_median_k_200.png}
                \caption{Obraz r�nicowy}
        \end{subfigure}
        \caption{Obrazy wynikowe dla sekwencji clip\_02.mpg}
\end{figure}

\newpage
Najgorsze warto�ci metryk zosta�y uzyskane dla sekwencji clip\_03.mpg i wynosz� one odpowiednio:


% Table generated by Excel2LaTeX from sheet 'Sheet1'
\begin{table}[htbp]
  \centering
    \begin{tabular}{|r|r|}
    \hline
    MSE   & 201,5417845 \bigstrut\\
    \hline
    PP(80\%) & 0,984985952 \bigstrut\\
    \hline
    PP(90\%) & 0,846380014 \bigstrut\\
    \hline
    PP(98\%) & 0,090032098 \bigstrut\\
    \hline
    PP(99\%) & 0,037495977 \bigstrut\\
    \hline
    \end{tabular}
  \caption{Warto�ci metryk dla sekwencji clip\_03.mpg}
\end{table}

Poni�ej przedstawiono obrazy: referencyjny, wynikowy oraz ich r�nic� dla tej sekwencji.


\begin{figure}[ht]
        \centering
        \begin{subfigure}[b]{0.32\textwidth}
                \centering
                \includegraphics[width=\textwidth]{clip_03_ref_gray.png}
                \caption{Obraz referencyjny}
        \end{subfigure}
        \begin{subfigure}[b]{0.32\textwidth}
                \centering
                \includegraphics[width=\textwidth]{bg_clip_03mpg_median_k_200.png}
                \caption{Obraz wynikowy}
        \end{subfigure}
        \begin{subfigure}[b]{0.32\textwidth}
                \centering
                \includegraphics[width=\textwidth]{bg_clip_03mpg_median_k_200.png}
                \caption{Obraz r�nicowy}
        \end{subfigure}
        \caption{Obrazy wynikowe dla sekwencji clip\_03.mpg}
\end{figure}


%------------------------------------------------------------------------------
\newpage
\section{Metoda aproksymacji �redniej z bufora ramek}

Dla tej metody najlepsze warto�ci metryk zosta�y uzyskane dla sekwencji fountain.mpg przy parametrze $\alpha = 0.9$ i wynosz� one odpowiednio:

% Table generated by Excel2LaTeX from sheet 'Sheet1'
\begin{table}[ht]
  \centering
    \begin{tabular}{|r|r|}
    \hline
    MSE   & 13,09215802 \bigstrut\\
    \hline
    PP(80\%) & 0,995739976 \bigstrut\\
    \hline
    PP(90\%) & 0,980837264 \bigstrut\\
    \hline
    PP(98\%) & 0,932871462 \bigstrut\\
    \hline
    PP(99\%) & 0,911379717 \bigstrut\\
    \hline
    \end{tabular}%
 \caption{Warto�ci metryk dla sekwencji fountain.mpg}
\end{table}%


Poni�ej przedstawiono obrazy: referencyjny, wynikowy oraz ich r�nic� dla tej sekwencji.


\begin{figure}[ht]
        \centering
        \begin{subfigure}[b]{0.32\textwidth}
                \centering
                \includegraphics[width=\textwidth]{fountain_ref_gray.png}
                \caption{Obraz referencyjny}
        \end{subfigure}
        \begin{subfigure}[b]{0.32\textwidth}
                \centering
                \includegraphics[width=\textwidth]{bg_fountainmpg_alpha_a099_k_200.png}
                \caption{Obraz wynikowy}
        \end{subfigure}
        \begin{subfigure}[b]{0.32\textwidth}
                \centering
                \includegraphics[width=\textwidth]{bg_fountainmpg_alpha_a099_k_200.png}
                \caption{Obraz r�nicowy}
        \end{subfigure}
        \caption{Obrazy wynikowe dla sekwencji fountain.mpg}
\end{figure}

\newpage
Najgorsze warto�ci metryk zosta�y uzyskane dla sekwencji clip\_03.mpg przy parametrze $\alpha = 0.8$ i wynosz� one odpowiednio:


\begin{table}[htbp]
  \centering
    \begin{tabular}{|r|r|}
    \hline
    MSE   & 208,8327099 \bigstrut\\
    \hline
    PP(80\%) & 0,979836289 \bigstrut\\
    \hline
    PP(90\%) & 0,66894719 \bigstrut\\
    \hline
    PP(98\%) & 0,085669674 \bigstrut\\
    \hline
    PP(99\%) & 0,038291913 \bigstrut\\
    \hline
    \end{tabular}%
  \caption{Warto�ci metryk dla sekwencji clip\_03.mpg}
\end{table}

Poni�ej przedstawiono obrazy: referencyjny, wynikowy oraz ich r�nic� dla tej sekwencji.


\begin{figure}[ht]
        \centering
        \begin{subfigure}[b]{0.32\textwidth}
                \centering
                \includegraphics[width=\textwidth]{clip_03_ref_gray.png}
                \caption{Obraz referencyjny}
        \end{subfigure}
        \begin{subfigure}[b]{0.32\textwidth}
                \centering
                \includegraphics[width=\textwidth]{bg_clip_03mpg_alpha_a099_k_200.png}
                \caption{Obraz wynikowy}
        \end{subfigure}
        \begin{subfigure}[b]{0.32\textwidth}
                \centering
                \includegraphics[width=\textwidth]{bg_clip_03mpg_alpha_a099_k_200.png}
                \caption{Obraz r�nicowy}
        \end{subfigure}
        \caption{Obrazy wynikowe dla sekwencji clip\_03.mpg}
\end{figure}

%------------------------------------------------------------------------------
\newpage
\section{Metoda Mixture of Gaussians}

Dla tej metody najlepsze warto�ci metryk zosta�y uzyskane dla sekwencji clip\_02.mpg i wynosz� one odpowiednio:

\begin{table}[htbp]
  \centering
    \begin{tabular}{|r|r|}
    \hline
    MSE   & 25,19326021 \bigstrut\\
    \hline
    PP(80\%) & 0,990661888 \bigstrut\\
    \hline
    PP(90\%) & 0,9738298 \bigstrut\\
    \hline
    PP(98\%) & 0,832057516 \bigstrut\\
    \hline
    PP(99\%) & 0,586496055 \bigstrut\\
    \hline
    \end{tabular}%
 \caption{Warto�ci metryk dla sekwencji fountain.mpg}
\end{table}%


Poni�ej przedstawiono obrazy: referencyjny, wynikowy oraz ich r�nic� dla tej sekwencji.


\begin{figure}[ht]
        \centering
        \begin{subfigure}[b]{0.32\textwidth}
                \centering
                \includegraphics[width=\textwidth]{clip_02_ref_gray.png}
                \caption{Obraz referencyjny}
        \end{subfigure}
        \begin{subfigure}[b]{0.32\textwidth}
                \centering
                \includegraphics[width=\textwidth]{clip_02_mog200.png}
                \caption{Obraz wynikowy}
        \end{subfigure}
        \begin{subfigure}[b]{0.32\textwidth}
                \centering
                \includegraphics[width=\textwidth]{clip_02_mog200.png}
                \caption{Obraz r�nicowy}
        \end{subfigure}
        \caption{Obrazy wynikowe dla sekwencji clip\_02.mpg}
\end{figure}

\newpage
Najgorsze warto�ci metryk zosta�y uzyskane dla sekwencji clip\_03.mpg przy parametrze $\alpha = 0.8$ i wynosz� one odpowiednio:


\begin{table}[htbp]
  \centering
    \begin{tabular}{|r|r|}
    \hline
    MSE   & 208,8327099 \bigstrut\\
    \hline
    PP(80\%) & 0,979836289 \bigstrut\\
    \hline
    PP(90\%) & 0,66894719 \bigstrut\\
    \hline
    PP(98\%) & 0,085669674 \bigstrut\\
    \hline
    PP(99\%) & 0,038291913 \bigstrut\\
    \hline
    \end{tabular}%
  \caption{Warto�ci metryk dla sekwencji clip\_03.mpg}
\end{table}

Poni�ej przedstawiono obrazy: referencyjny, wynikowy oraz ich r�nic� dla tej sekwencji.


\begin{figure}[ht]
        \centering
        \begin{subfigure}[b]{0.32\textwidth}
                \centering
                \includegraphics[width=\textwidth]{clip_03_ref_gray.png}
                \caption{Obraz referencyjny}
        \end{subfigure}
        \begin{subfigure}[b]{0.32\textwidth}
                \centering
                \includegraphics[width=\textwidth]{clip_03_mog200.png}
                \caption{Obraz wynikowy}
        \end{subfigure}
        \begin{subfigure}[b]{0.32\textwidth}
                \centering
                \includegraphics[width=\textwidth]{clip_03_mog200.png}
                \caption{Obraz r�nicowy}
        \end{subfigure}
        \caption{Obrazy wynikowe dla sekwencji clip\_03.mpg}
\end{figure}
\chapter{Por�wnania metod}

W niniejszym rozdziale zostan� por�wnane mi�dzy sob� niekt�re metody.
W celu przeprowadzenia por�wnania dokonano tabelaryzacji najlepszych dla danej metody wynik�w \ref{tabelaporownawcza} oraz przedstawiono je na wykresie por�wnawczym \ref{wykresporownawczy}.

\begin{table}[H]
  \centering
    \begin{tabular}{|c|r|r|r|r|r|r|}
    \hline
    \multirow{2}[4]{*}{\textbf{Sekwencja}} & \multicolumn{6}{c|}{\textbf{Metoda}} \bigstrut\\
\cline{2-7}          & \multicolumn{1}{c|}{\textbf{�rednia}} & \multicolumn{1}{c|}{\textbf{Mediana}} & \multicolumn{1}{c|}{\textbf{MOG}} & \multicolumn{1}{c|}{\textbf{\parbox[top][2cm][c]{2.3cm}{Aproksymacja �redniej}}} & \multicolumn{1}{c|}{\textbf{DCT}} & \multicolumn{1}{c|}{\textbf{Hadamard}} \bigstrut\\
    \hline
    \textbf{clip\_01.mpg} & 85.9989 & 53.8921 & 86.2640 & 64.5492 & 43.1881 & 43.1868 \bigstrut\\
    \hline
    \textbf{clip\_02.mpg} & 38.3934 & 13.1536 & 25.1933 & 29.5430 & 37.5261 & 37.5051 \bigstrut\\
    \hline
    \textbf{clip\_03.mpg} & 198.5494 & 201.5418 & 181.0108 & 151.6200 & 183.3507 & 183.3458 \bigstrut\\
    \hline
    \textbf{clip\_04.mpg} & 62.8719 & 32.7241 & 63.4837 & 21.6948 & 34.1747 & 34.3609 \bigstrut\\
    \hline
    \textbf{fountain.mpg} & 37.7537 & 36.5344 & 29.1133 & 13.0922 & 70.7695 & 70.7736 \bigstrut\\
    \hline
    \textbf{highway.mpg} & 56.4986 & 52.7896 & 55.6217 & 71.3761 & 47.3152 & 47.8593 \bigstrut\\
    \hline
    \textbf{pedestrians.mpg} & 105.0423 & 24.9158 & 33.3411 & 67.8489 & 84.5606 & 84.5457 \bigstrut\\
    \hline
    \end{tabular}%
\caption{Wykres por�wnawczy wszystkich metod}
\label{tabelaporownawcza}
\end{table}%



\begin{figure}[H]
\includegraphics[width=\textwidth]{wszyetkiewykres.PNG}
\caption{Wykres por�wnawczy wszystkich metod}
\label{wykresporownawczy}
\end{figure}
\section{Por�wnanie metody MOG z metod� mediany z bufora}

% Table generated by Excel2LaTeX from sheet 'Sheet1'
\begin{table}[htbp]
  \centering
    \begin{tabular}{|r|r|r|}
    \hline
    \multicolumn{1}{|c|}{\multirow{2}[4]{*}{\textbf{Sekwencja}}} & \multicolumn{2}{c|}{\textbf{Metoda}} \bigstrut\\
\cline{2-3}    \multicolumn{1}{|c|}{} & \textbf{Mediana} & \textbf{MOG} \bigstrut\\
    \hline
    \textbf{clip\_01.mpg} & 53,8921 & 86,2640 \bigstrut\\
    \hline
    \textbf{clip\_02.mpg} & 13,1536 & 25,1933 \bigstrut\\
    \hline
    \textbf{clip\_03.mpg} & 201,5418 & 181,0108 \bigstrut\\
    \hline
    \textbf{clip\_04.mpg} & 32,7241 & 63,4837 \bigstrut\\
    \hline
    \textbf{fountain.mpg} & 36,5344 & 29,1133 \bigstrut\\
    \hline
    \textbf{highway.mpg} & 52,7896 & 55,6217 \bigstrut\\
    \hline
    \textbf{pedestrians.mpg} & 24,9158 & 33,3411 \bigstrut\\
    \hline
    \end{tabular}%
\caption{Tabela por�wnawcza metod MOG i mediany z bufora}
\label{MOGMEDporownanietabela}
\end{table}%

\begin{figure}[H]
\includegraphics[width=\textwidth]{MOGMED.PNG}
\caption{Wykres por�wnawczy metody MOG z metod� mediany z bufora}
\label{wykresporownawczyMOGMED}
\end{figure}

\chapter{Wnioski}

Po szczeg�owym przeanalizowaniu ka�dej metody na 7 sewencjach testowych mo�na doj�� do wniosku, i� �adnej z nich nie mo�na okre�li� mianem uniwersalnej - rezultaty zale�� od czynnik�w takich jak o�wietlenie, wyst�powanie cienia, dynamika sekwencji wideo, przej�cia tonalne element�w t�a. Dodatkowo, ka�da z metod posiada jeden lub kilka wsp�czynnik�w, kt�re nale�y dobra� indywidualnie dla ka�dej sekwencji (cz�sto metod� pr�b i b��d�w). Ich z�y dob�r mo�e doprowadzi� do niepoprawnego zainicjalizowania t�a i powstania b��dnych modeli, co przedstawiono na rysunku \ref{fig:zlemodele}.


\begin{figure}[H]
        \centering
        \begin{subfigure}[b]{0.3\textwidth}
                \centering
                \includegraphics[width=\textwidth]{clip_02_broken.png}
                \centering
                \caption{$T_{corr}=0.8$ $T_{MAD}=10$}
        \end{subfigure}
        \begin{subfigure}[b]{0.3\textwidth}
                \centering
                \includegraphics[width=\textwidth]{fountain_broken.png}
                \caption{$T_{corr}=0.9$ $T_{MAD}=10$}        
                \end{subfigure}
        \begin{subfigure}[b]{0.3\textwidth}
                \centering
                \includegraphics[width=\textwidth]{highway_broken.png}
                \caption{$T_{corr}=0.9$ $T_{MAD}=10$}
        \end{subfigure}
        \caption{B��dy w wynikach dzia�ania algorytmu przestrzennego}
        \label{fig:zlemodele}
\end{figure}

Du�ym zaskoczeniem okaza�y si� wyniki dzia�ania algorytm�w przestrzennych: w wielu przypadkach skuteczniejsze okazywa�y si� proste metody pikselowe (np. mediana w przypadku sekwencji (clip\_02.mpg, clip\_04.mpg, fountain.mpg, pedestrians.mpg), kt�re ponadto inicjalizowa�y t�o znacznie szybciej. Dodatkowo, specyfika tych metod powoduje, i� w niekt�rych przypadkach (gdy istnieje wiele kandydat�w dla ka�dego bloku) t�o nie zostanie ca�kowicie zainicjalizowane - sta�o si� tak dla ka�dej sekwencji w przypadku ustawienia zbyt wysokich warto�ci parametru $T_{corr}$ oraz zbyt niskich warto�ci $T_{MAD}$. Efekt ten mo�na spr�bowa� zminimalizowa� poprzez wprowadzenie nast�puj�cych ulepsze�:
\begin{itemize}
\item rekonstrukcj� brakuj�cych blok�w kandydatem z najwy�sz� wag� - jest to modyfikacja, kt�ra nie wyd�u�a znacz�co czasu dzia�ania algorytmu - wykonuje on tylko jedn� dodatkow� iteracj� w~ostatniej fazie. Dzi�ki tej modyfikacji mo�na uzyska� zawsze w pe�ni zrekonstruowane t�o, jednak�e, nie zawsze b�dzie ono poprawne - blok z najwy�sz� wag� nie zawsze bowiem musi by� najlepiej pasuj�cym fragmentem t�a - dzieje si� tak np. wtedy, gdy przez d�u�szy okres czasu trwania sekwencji t�o zas�ania obiekt, kt�ry w pewnym momencie znika (np. samoch�d) - wtedy pomimo tego, i� posiada on wy�sz� wag�, lepiej dopasowany do s�siednich blok�w b�dzie ods�oni�ty po odjechaniu pojazdu fragment jezdni.

\item znalezienie superbloku, gdzie istniej� ju� 2 zrekonstruowane fragmenty t�a, przyj�cie za fragment t�a w 3 bloku kandydata z najwy�sz� wag� i uruchomienie algorytmu na nowo - ulepszenie to powinno da� znacznie lepsze rezultaty, jednak�e zwi�ksza znacznie z�o�ono�� obliczeniow�. Wyst�puje tu r�wnie� takie samo ryzyko z wyborem kandydata jak w poprzedniej propozycji, jednak�e jest ono znacznie mniejsze.
\end{itemize}

Ulepszeniu mo�na podda� r�wnie� metodologi� doboru ramek dla ka�dego z wymienionych algorytm�w. Dobrym pomys�em wydaje si� by� wybieranie ramek w pewnych odst�pach czasu, co spowoduje zwi�kszenie prawdopodobie�stwa zaobserwowania t�a w danym miejscu.

Spo�r�d wszystkich przetestowanych algorytm�w metoda wykorzystuj�ca aproksymacj� �redniej da�a najgorsze wyniki b��du �redniokwadratowego, niezale�nie od parametru alfa, a analiza wizualna efekt�w jej dzia�ania pozwala stwierdzi�, i� nie nadaje si� ona ani do inicjalizacji, ani do generacji t�a. Najlepsze wyniki uzyska�y metody MOG i mediana, oferuj�c jednocze�nie do�� kr�tki czas oblicze� (ok. 20 sekund w przypadku Mixture of Gaussian i 6 minut dla mediany). Metody przestrzenne r�wnie� uzyska�y do�� dobre rezultaty, jednak�e czas potrzebny na zainicjalizowanie przez nie t�a okaza� si� wielokrotnie d�u�szy (rz�du 3 godzin na sekwencj� 200 ramek). Bior�c jednak pod uwag�, �e s� one stosunkowo nowe (zosta�y one zaproponowane w artyku�ach, kt�re opublikowano odpowiednio w roku 2009 dla metody wykorzystuj�cej transformat� DCT oraz w 2010 dla podej�cia z wykorzystaniem transformaty Hadamarda) mo�na domniemywa�, �e ich kolejne wersje b�d� coraz skuteczniejsze i szybsze. 
\chapter{S�ownik u�ytych poj��}
\label{slownik}

\begin{description}
\item[inicjalizacja t�a] proces jednorazowy, w kt�rym na podstawie pewnej liczby ramek i odpowiednich algorytm�w zostaje uzyskany model t�a
\item[generacja t�a] ci�g�y proces aktualizacji wst�pnie zainicjalizowanego modelu  t�a
\item[blok] wydzielona kwadratowa cze�� ramki o~rozmiarach $N\times N$, zawieraj�ca opr�cz pikseli informacj� o~wadze superbloku
\item[superblok] grupa 4~s�siaduj�cych ze sob� blok�w, tworz�ca wi�kszy blok (superblok) o~rozmiarach $2N~\times~2N$. Przyk�adowy superblok zosta� przedstawiony na rysunku \ref{slowniSUPERBLOK_IMG}.
\begin{figure}[ht]
\begin{center}
\includegraphics{blok.png}
\caption{Blok X~oraz jego otoczenie, tworz�ce razem z~nim superblok}
\label{slowniSUPERBLOK_IMG}
\end{center}
\end{figure}  
\item[grupa kandydat�w] utrzymywana dla ka�dego bloku grupa blok�w, kt�re z pewnym prawdopodobie�stwem mog� by� fragmentem t�a
\item[lokalizacja blokowa] - para wsp�rz�dnych (x,y), wyznaczaj�ca miejsce w sekwencji wideo, gdzie znajduje si� dany blok
\end{description}

% itd.
% \appendix
% \chapter{Dodatek A - spis zawarto�ci p�yty CD}

Do niniejszej pracy zosta�a do��czona p�yta CD zawieraj�ca foldery:
\begin{description}
\item[praca] - praca w formacie pdf
\item[praca\_latex] - praca w formacie \LaTeX
\item[sekwencje] - sekwencje u�yte do bada�
\subitem clip\_01.mpg
\subitem clip\_02.mpg
\subitem clip\_03.mpg
\subitem clip\_04.mpg
\subitem fountain.mpg
\subitem highway.mpg
\subitem pedestrians.mpg
\item[kod] - kody �r�d�owe program�w napisanych w ramach pracy
\end{description}
% \chapter{Dodatek B - opis informatyczny programu}

W ramach niniejszej pracy zosta�y wykonane 2 programy napisane w j�zyku C++ przy u�yciu biblioteki OpenCV.

Pierwszy z nich \textbf{robust.exe} zawiera w sobie implementacj� 2 algorytm�w. S� nimi algorytm MOG oraz blokowy, kt�ry w zale�no�ci od parametr�w wej�ciowych u�ywa transformaty Hadamarda lub DCT. 

Program ten przyjmuje parametry wej�ciowe w nast�puj�cej kolejno�ci:

\begin{enumerate}
\item wielko�� bufora ramek
\item typ transformaty
\item nie wiem co wi�cej
\end{enumerate}

Drugi program \textbf{simple\_pixel\_methods.exe} zawiera w sobie implementacj� 3 algorytm�w. S� nimi algorytmy: �redniej z bufora, aproksymacji �redniej z bufora oraz mediany.

Program ten przyjmuje parametry wej�ciowe w nast�puj�cej kolejno�ci:

\begin{enumerate}
\item wielko�� bufora ramek
\item typ metody (0 - aproksymacja, 1 - �rednia, 2 - mediana)
\item warto�� parametru $\alpha$ w przypadku wybrania metody aproksymacji
\end{enumerate}

nie wiem co tu dalej napisa�...
% itd.

\bibliographystyle{alpha}
\bibliography{bibliografia}
%\begin{thebibliography}{1}
%
%\bibitem{Dil00}
%A.~Diller.
%\newblock {\em LaTeX wiersz po wierszu}.
%\newblock Wydawnictwo Helion, Gliwice, 2000.
%
%\bibitem{Lam92}
%L.~Lamport.
%\newblock {\em LaTeX system przygotowywania dokument�w}.
%\newblock Wydawnictwo Ariel, Krakow, 1992.
%
%\bibitem{Alvis2011}
%M.~Szpyrka.
%\newblock {\em {On Line Alvis Manual}}.
%\newblock AGH University of Science and Technology, 2011.cccccc
%\newblock \\\texttt{http://fm.ia.agh.edu.pl/alvis:manual}.
%
%\end{thebibliography}

\end{document}
