\chapter{Metodologia bada�}
\section{U�yte sekwencje}
W niniejszej pracy do oceny jako�ci algorytm�w inicjalizacji t�a u�yto pierwszych 200 ramek nast�puj�cych sekwencji video:
\begin{enumerate}
\item clip\_01.mpg - sekwencja zarejestrowana przez kamer� umieszczon� na �cianie budynku C3 AGH w pochmurny dzie�, przedstawiaj�ca ludzi chodz�cych po chodniku oraz cz�� jezdni ulicy Czarnowiejskiej wraz z poruszaj�cymi si� na niej samochodami
\item clip\_02.mpg - jak w pkt. 1, lecz sekwencja jest w gorszej jako�ci
\item clip\_03.mpg - jak w pkt. 1, lecz przy zmiennych warunkach o�wietleniowych
\item clip\_04.mpg - jak w pkt. 1, lecz ze sta�ym cieniem
\item fountain.mpg - sekwencja przedstawiaj�ca na pierwszym planie dzia�aj�c� fontann� na tle parkingu z poruszaj�cymi si� samochodami
\item highway.mpg - sekwencja zarejestrowana przez kamer� umieszczon� nad autostrad�
\item pedestrians.mpg  - sekwencja zarejestrowana przez kamer� umieszczon� na dworcu kolejowym, przedstawiaj�ca poruszaj�cych si� po peronie pasa�er�w
\item sidewalk.mpg - sekwencja zarejestrowana przez kamer� umieszczon� nad przej�ciem dla pieszych; na obrazie wyst�puj� mocne drgania kamery.
\end{enumerate}

\section{U�yte metryki}


