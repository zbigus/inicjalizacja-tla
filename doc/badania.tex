\chapter{Badania}
\label{badania}
Dla ka�dej z u�ytych sekwencji wideo zostan� przedstawione w postaci tabel dwie metryki por�wnawcze: MSE - b��d �redniokwadratowy oraz PP - Procent podobie�stwa o zadanym progu, przy czym przyj�ty pr�g zostanie podany w nawiasach. Dla ka�dej metody, dla wybranych wsp�czynnik�w j� opisuj�cych, przedstawione b�d�: obraz r�nicowy, histogram oraz histogram r�nicowy.  
\newpage
\section{Sekwencja clip\_01.mpg}

\begin{table}[ht]
\begin{tabular}{|*{14}{l|}}
\cline{1-13}
 \multirow{2}{*}{metryka} &
 \multirow{2}{*}{�rednia}& \multirow{2}{*}{mediana} &
 \multirow{2}{*}{MOG} & \multicolumn{9}{|c|}{aproksymacja �redniej przy u�yciu parametru $\alpha$} \\
 \cline{5-14}
 & & & & 0.1 & 0.2 & 0.3 & 0.4 & 0.5 & 0.6 & 0.7 & 0.8 & 0.9 & wsp�czynnik $\alpha$ \\
 \hline
 MSE \\ 
 \cline{1-13}
 PP (80 \%) \\ 
 \cline{1-13}
 PP (90 \%) \\ 
 \cline{1-13}
 PP (98 \%) \\ 
 \cline{1-13}
 PP (99 \%) \\ 
  \cline{1-13}
\end{tabular}
\caption{Warto�ci metryk por�wnawczych dla metod pikselowych - sekwencja clip\_01.mpg}
\end{table}
\begin{table}[ht]
\begin{tabular}{|*{10}{l|}}
\cline{1-9}
  \multirow{3}{*}{metryka} &
  \multicolumn{8}{|c|}{transformata DCT} \\
\cline{2-10}
 & \multicolumn{4}{|c|}{10} & \multicolumn{4}{|c|}{20} & $T_{mad}$ \\
 \cline{2-10}
 & 0.6 & 0.7 & 0.8 & 0.9 & 0.6 & 0.7 & 0.8 & 0.9 & $T_{corr}$ \\
\hline
MSE & 6.73 & 6.90 & 7.37 & 8.00 & 7.40 & 7.55 & 8.12 \\ 
\cline{1-9}
PP (80 \% ) & 99.94 & 99.96 & 99.84 & 99.82 & 99.74 & 99.76 & 99.70\\
\cline{1-9}
PP (90 \% ) & 99.77 & 99.75 & 99.64 & 99.56 & 99.31 & 99.34 & 99.26\\
\cline{1-9}
PP (98 \% ) & 94.18 & 94.06 & 93.78 & 93.38 & 94.11 & 93.96 & 93.59\\
\cline{1-9}
PP (99 \% ) & 84.73& 84.48 & 84.06 & 83.30 & 85.25 & 84.83& 84.13\\
\cline{1-9}
\end{tabular}
\caption{Warto�ci metryk por�wnawczych dla metody z transformat� DCT - sekwencja clip\_01.mpg}
\end{table}
\begin{table}[ht]
\begin{tabular}{|*{10}{l|}}
\cline{1-9}
  \multirow{3}{*}{metryka} &
  \multicolumn{8}{|c|}{transformata Hadamarda} \\
\cline{2-10}
 & \multicolumn{4}{|c|}{10} & \multicolumn{4}{|c|}{20} & $T_{mad}$ \\
 \cline{2-10}
 & 0.6 & 0.7 & 0.8 & 0.9 & 0.6 & 0.7 & 0.8 & 0.9 & $T_{corr}$ \\
\hline
MSE & 6.73 & 6.90 & 7.37 & 8.00 & 7.40 & 7.55 & 8.12 \\ 
\cline{1-9}
PP (80 \% ) & 99.94 & 99.96 & 99.84 & 99.82 & 99.74 & 99.76 & 99.70\\
\cline{1-9}
PP (90 \% ) & 99.77 & 99.75 & 99.64 & 99.56 & 99.31 & 99.34 & 99.26\\
\cline{1-9}
PP (98 \% ) & 94.18 & 94.06 & 93.78 & 93.38 & 94.11 & 93.96 & 93.59\\
\cline{1-9}
PP (99 \% ) & 84.73& 84.48 & 84.06 & 83.30 & 85.25 & 84.83& 84.13\\
\cline{1-9}
\end{tabular}
\caption{Warto�ci metryk por�wnawczych dla metody z transformat� Hadamarda - sekwencja clip\_01.mpg}
\end{table}